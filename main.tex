\documentclass{article}
\usepackage[utf8]{inputenc}
\usepackage[spanish]{babel}
\usepackage{listings}
\usepackage{graphicx}
\graphicspath{ {images/} }
\usepackage{cite}

\begin{document}

\begin{titlepage}
    \begin{center}
        \vspace*{1cm}
            
        \Huge
        \textbf{Parcial 1}
            
        \vspace{0.5cm}
        \LARGE
        Calistenia
            
        \vspace{1.5cm}
            
        \textbf{Deizon Arley Guarin Marin}
            
        \vfill
            
        \vspace{0.8cm}
            
        \Large
        Despartamento de Ingeniería Electrónica y Telecomunicaciones\\
        Universidad de Antioquia\\
        Medellín\\
        Marzo de 2021
            
    \end{center}
\end{titlepage}


\newpage
\textbf{Pasos de la actividad}\\
estando dos targetas del mismo tamaño y una hoja de papel sobre estas y a su vez
estas dos apoyadas en una superficie horizontal y plana se deben seguir los siguientes pasos(estos pasos se deben hacer con una sola mano):\\
1. levantar la hoja suavemente y colocarla a un lado\\
2. definir la mitad de hoja sin doblarla\\
3. definir el lado mas grande y mas pequeño de las targetas  
3. levantar las dos targetas(con una sola mano)cogiendolas por el lado mas grande
4.poner las targetas por el lado mas pequeño sobre la mitad de la hoja parandolas de forma vertical\\
5. la parte de la targeta que esta en contacto conla hoja con un dedo irlas separando poco a poco, la parte de encima de las dos targetas debe permanecer unida hasta formar un triangulo\\
6. ir soltando poco a poco\\
7. si las targetas no mantienen el equilibrio volver  a usar los pasos desde el paso tres\\

\textbf{conclusion de la actividad}\\
-tiende a ser demisiado ambigua 
-es dificil de ser lo soficientemente preciso
-las personas pueden interpretarlo de diferente manera
\end{document}
